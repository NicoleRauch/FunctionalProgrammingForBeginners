%%%%%%%%%%%%%%%%%%%%%%%%%%%%%%%%%%%%%%%%%%%%%%%%%%
\begin{frame}[fragile]{Bekannte funktionale Sprachen}
x
\end{frame}

%%%%%%%%%%%%%%%%%%%%%%%%%%%%%%%%%%%%%%%%%%%%%%%%%%
\begin{frame}[fragile]{Warum eigentlich \glqq{}funktional\grqq{}?}

\begin{center}
\Large
Funktionen sind \glqq{}first order citizens\grqq{}
\end{center}

\vspace{2em}

\begin{center}
Mit Funktionen kann man dasselbe machen wie mit Strings oder Zahlen
\end{center}

\end{frame}

%%%%%%%%%%%%%%%%%%%%%%%%%%%%%%%%%%%%%%%%%%%%%%%%%%
\begin{frame}[fragile]{Funktionen können Variablen zugewiesen werden}

\begin{lstlisting}{Haskell}
var times = function (x, y) { return x * y; }

expect(times(3, 5)).toEqual(15);
\end{lstlisting}

\begin{lstlisting}{Haskell}
times (x, y) = x * y

times (3,5) == 15
\end{lstlisting}

\end{frame}

%%%%%%%%%%%%%%%%%%%%%%%%%%%%%%%%%%%%%%%%%%%%%%%%%%
\begin{frame}[fragile]{Funktionen können als Funktionsparameter übergeben werden}

\begin{lstlisting}{javascript}
var apply = function (func, arg) { return func(arg); }

var times3 = function (y) { return 3 * y; };

expect(apply(times3, 5)).toEqual(15);
\end{lstlisting}

\end{frame}

%%%%%%%%%%%%%%%%%%%%%%%%%%%%%%%%%%%%%%%%%%%%%%%%%%
\begin{frame}[fragile]{Funktionen können von Funktionen zurückgegeben werden}

\begin{lstlisting}{javascript}
var times = function (x) { return function (y) { return x * y; }; }

expect(times(3)(5)).toEqual(15);
\end{lstlisting}

\end{frame}

%%%%%%%%%%%%%%%%%%%%%%%%%%%%%%%%%%%%%%%%%%%%%%%%%%
\begin{frame}[fragile]{Komisch, oder?}

\begin{lstlisting}{javascript}
var times = function (x, y) { return x * y; }
expect(times(3, 5)).toEqual(15);

var times = function (x) { return function (y) { return x * y; }; }
expect(times(3)(5)).toEqual(15);
\end{lstlisting}

currying
\end{frame}

%%%%%%%%%%%%%%%%%%%%%%%%%%%%%%%%%%%%%%%%%%%%%%%%%%
\begin{frame}[fragile]{Noch mehr lustige Dinge}

\begin{lstlisting}{javascript}
\end{lstlisting}

closures
\end{frame}

%%%%%%%%%%%%%%%%%%%%%%%%%%%%%%%%%%%%%%%%%%%%%%%%%%
\begin{frame}[fragile]{Und wenn ich nichts zurückgeben will?}

\begin{lstlisting}{javascript}
\end{lstlisting}

Unit
\end{frame}

%%%%%%%%%%%%%%%%%%%%%%%%%%%%%%%%%%%%%%%%%%%%%%%%%%
\begin{frame}[fragile]{Wichtige Bibliotheksfunktionen: filter}
\begin{itemize}
\item filter oder auch select
\item Nimmt eine Collection und eine Funktion
\item Liefert diejenigen Elemente der Collection, für die die Funktion \texttt{true} ergibt
\end{itemize}

Beispiel in JavaScript:
\begin{lstlisting}{javascript}
var result = _.filter([1, 2, 3, 4], function (x) { return x % 2 === 0; });
    
expect(result).toEqual([2, 4]);
\end{lstlisting}
\vfill
(mit freundlicher Unterstützung der lodash-Bibliothek)
\end{frame}

% find oder detect

%%%%%%%%%%%%%%%%%%%%%%%%%%%%%%%%%%%%%%%%%%%%%%%%%%
\begin{frame}[fragile]{Wichtige Bibliotheksfunktionen: map}
\begin{itemize}
\item map oder auch collect
\item Nimmt eine Collection und eine Funktion
\item Liefert eine Collection, in der die Funktion auf jedes Element der ursprünglichen Collection angewandt wurde
\end{itemize}

Beispiel in JavaScript:
\begin{lstlisting}{javascript}
var result = _.map( [1, 2, 3, 4], function (x) { return x + 5; } );

expect(result).toEqual([6, 7, 8, 9]);
\end{lstlisting}

\end{frame}

%%%%%%%%%%%%%%%%%%%%%%%%%%%%%%%%%%%%%%%%%%%%%%%%%%
\begin{frame}[fragile]{Wichtige Bibliotheksfunktionen: fold}
\begin{itemize}
\item fold oder auch reduce oder inject
\item Nimmt eine Collection und eine Funktion
\item Verbindet zwei Elemente der Collection mit Hilfe der Funktion zu einem Element
\item Verbindet das Ergebnis mit dem nächsten Element der Collection zu einem Element
\item Setzt dies für alle Elemente der Collection fort, bis nur noch ein Element übrigbleibt
\end{itemize}

Beispiel in JavaScript:
\begin{lstlisting}{javascript}
var result = _.foldl( [1, 2, 3, 4], function (x, y) { return x * y; } );

expect(result).toEqual(24);
\end{lstlisting}

\end{frame}

%%%%%%%%%%%%%%%%%%%%%%%%%%%%%%%%%%%%%%%%%%%%%%%%%%
\begin{frame}[fragile]{Eine einfache Berechnung}

\[
summe = \sum_{i=1}^{10}i^2
\]

\vspace{5em}

\begin{lstlisting}{java}
int summe = 0;
for(int i = 1; i <= 10; i++) {
  summe = summe + i * i;
}
\end{lstlisting}
\end{frame}

%%%%%%%%%%%%%%%%%%%%%%%%%%%%%%%%%%%%%%%%%%%%%%%%%%
\begin{frame}[fragile]{Exkurs: Clean Code}
Single Responsibility Principle
\\[2em]

Wie viele Verantwortlichkeiten hat dieser Code?
\begin{lstlisting}{java}
int summe = 0;
for(int i = 1; i <= 10; i++) {
  summe = summe + i * i;
}
\end{lstlisting}

\begin{itemize}
\item Erzeugen der Zahlenfolge von 1 bis 10
\item Quadrieren einer Zahl
\item Berechnen der Quadratzahl jeder Zahl in der Folge
\item Addieren zweier Zahlen
\item Aufsummieren der Quadratzahlen
\end{itemize}

\end{frame}

%%%%%%%%%%%%%%%%%%%%%%%%%%%%%%%%%%%%%%%%%%%%%%%%%%
\begin{frame}[fragile]{Trennen der Verantwortlichkeiten}
\begin{lstlisting}{javascript}
var zahlenfolge = _.range(1, 11);

var quadriere = function (i) { return i * i; };

var quadrierteFolge = _.map(zahlenfolge, quadriere)

var addiere = function (summe, summand) { return summe + summand; };

var summe = _.reduce(quadrierteFolge, addiere);
\end{lstlisting}
\end{frame}

%%%%%%%%%%%%%%%%%%%%%%%%%%%%%%%%%%%%%%%%%%%%%%%%%%
\begin{frame}[fragile]{Zusammensetzen der Komponenten}
\begin{lstlisting}{javascript}
var quadriere = function (i) { return i * i; };
var addiere = function (summe, summand) { return summe + summand; };
\end{lstlisting}

\begin{lstlisting}{javascript}
var summe = _.reduce(_.map(_.range(1, 11), quadriere), addiere);
\end{lstlisting}

oder

\begin{lstlisting}{javascript}
var summe = _(1).range(11).map(quadriere).reduce(addiere);
\end{lstlisting}

\end{frame}


%%%%%%%%%%%%%%%%%%%%%%%%%%%%%%%%%%%%%%%%%%%%%%%%%%
\begin{frame}[fragile]{Zweites Standbein: Rekursion}
von Funktionen

von Datentypen

\end{frame}

%%%%%%%%%%%%%%%%%%%%%%%%%%%%%%%%%%%%%%%%%%%%%%%%%%
\begin{frame}[fragile]{Rekursive Funktionen}
x
\end{frame}

%%%%%%%%%%%%%%%%%%%%%%%%%%%%%%%%%%%%%%%%%%%%%%%%%%
\begin{frame}[fragile]{Rekursive Datentypen}
x
\end{frame}

%%%%%%%%%%%%%%%%%%%%%%%%%%%%%%%%%%%%%%%%%%%%%%%%%%
\begin{frame}[fragile]{Pattern Matching}
x
\end{frame}


%%%%%%%%%%%%%%%%%%%%%%%%%%%%%%%%%%%%%%%%%%%%%%%%%%
{
\usebackgroundtemplate{\includegraphics[width=\paperwidth,height=\paperheight]{background-slide.png}}
\begin{frame}{}

        ~\\[4em]
        Code \& Folien auf GitHub:
        \begin{center}
                \url{https://github.com/NicoleRauch/FunctionalProgrammingForBeginners}
        \end{center}

        ~\\[1em]
        \begin{block}{Nicole Rauch}
        \begin{description}[Twitterxx]
                \item[E-Mail]  \href{mailto:nicole.rauch@msg-gillardon.de}{\texttt{nicole.rauch@msg-gillardon.de}}
                \item[Twitter] \href{http://twitter.com/NicoleRauch}{\texttt{@NicoleRauch}}
        \end{description}
        \end{block}
\end{frame}
}
