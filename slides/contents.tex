%%%%%%%%%%%%%%%%%%%%%%%%%%%%%%%%%%%%%%%%%%%%%%%%%%
\begin{frame}[fragile]{\de{Funktional, das ist doch nur für Esoteriker?!}\en{Functional, isn't that a totally esoteric subject?!}}

\onslide+<2>
\begin{itemize}
\de{
\item ABN AMRO Amsterdam \textcolor{gray}{\textit{Risikoanalysen Investmentbanking}}
\item AT\&T \textcolor{gray}{\textit{Netzwerksicherheit: Verarbeitung von Internet-Missbrauchsmeldungen}}
\item Bank of America Merril Lynch \\ \textcolor{gray}{\textit{Backend: Laden \& Transformieren von Daten}}
\item Barclays Capital Quantitative Analytics Group \\ \textcolor{gray}{\textit{Mathematische Modellierung von Derivaten}}
\item Bluespec, Inc. \textcolor{gray}{\textit{Modellierung \& Verifikation integrierter Schaltkreise}}
\item Credit Suisse \textcolor{gray}{\textit{Prüfen und Bearbeiten von Spreadsheets}}
\item Deutsche Bank Equity Proprietary Trading, Directional Credit Trading \textcolor{gray}{\textit{Gesamte Software-Infrastruktur}}
\item Facebook \textcolor{gray}{\textit{Interne Tools}}
\item Factis Research, Freiburg \textcolor{gray}{\textit{Mobil-Lösungen (Backend)}}
\item fortytools gmbh \textcolor{gray}{\textit{Webbasierte Produktivitätstools - REST-Backend}}
\item Functor AB, Stockholm \textcolor{gray}{\textit{Statische Codeanalyse}}
}
\en{
\item ABN AMRO Amsterdam \textcolor{gray}{\textit{Risk analysis in investment banking}}
\item AT\&T \textcolor{gray}{\textit{Network security: processing of internet abuse complaints}}
\item Bank of America Merril Lynch \\ \textcolor{gray}{\textit{Backend data transformation and loading}}
\item Barclays Capital Quantitative Analytics Group \\ \textcolor{gray}{\textit{Mathematical modelling of equity derivatives}}
\item Bluespec, Inc. \textcolor{gray}{\textit{Modelling \& verification of integrated circuits}}
\item Credit Suisse \textcolor{gray}{\textit{Checking, manipulating and transforming spreadsheets}}
\item Deutsche Bank Equity Proprietary Trading, Directional Credit Trading \textcolor{gray}{\textit{All its software infrastructure}}
\item Facebook \textcolor{gray}{\textit{Internal tools}}
\item Factis Research, Freiburg \textcolor{gray}{\textit{Mobile solutions (backend)}}
\item fortytools gmbh \textcolor{gray}{\textit{web-based productivity tools - REST-backend}}
\item Functor AB, Stockholm \textcolor{gray}{\textit{static analysis}}
}
\end{itemize}
\end{frame}

\begin{frame}[fragile]{\de{Funktional, das ist doch nur für Esoteriker?!}\en{Functional, isn't that a totally esoteric subject?!}}
\begin{itemize}
\de{
\item Galois, Inc \textcolor{gray}{\textit{Security, Informationssicherheit, Kryptographie}}
\item Google \textcolor{gray}{\textit{Interne Projekte}}
\item IMVU, Inc. \textcolor{gray}{\textit{Social entertainment}}
\item JanRain \textcolor{gray}{\textit{Netzwerk- und Web-Software}}
\item MITRE \textcolor{gray}{\textit{Analyse kryptographischer Protokolle}}
\item New York Times \textcolor{gray}{\textit{Bildverarbeitung für die New York Fashion Week}}
\item NVIDIA \textcolor{gray}{\textit{In-House Tools}}
\item Parallel Scientific \textcolor{gray}{\textit{Hochskalierbares Cluster-Verwaltungssystem}}
\item Sankel Software \textcolor{gray}{\textit{CAD/CAM, Spiele, Computeranimation}}
\item Silk, Amsterdam \textcolor{gray}{\textit{Filtern und Visualisieren großer Datenmengen}}
\item Skedge Me \textcolor{gray}{\textit{Online-Terminvereinbarungen}}
\item Standard Chartered \textcolor{gray}{\textit{Bankensoftware}}
\item Starling Software, Tokio \textcolor{gray}{\textit{Automatisiertes Optionshandelssystem}}
\item Suite Solutions \textcolor{gray}{\textit{Verwaltung technischer Dokumentationen}}
}
\en{
\item Galois, Inc \textcolor{gray}{\textit{Security, information assurance and cryptography}}
\item Google \textcolor{gray}{\textit{Internal projects}}
\item IMVU, Inc. \textcolor{gray}{\textit{Social entertainment}}
\item JanRain \textcolor{gray}{\textit{Network and web software}}
\item MITRE \textcolor{gray}{\textit{Analysis of kryptographic protocols}}
\item New York Times \textcolor{gray}{\textit{Image processing for the New York Fashion Week}}
\item NVIDIA \textcolor{gray}{\textit{In-house tools}}
\item Parallel Scientific \textcolor{gray}{\textit{High-availability cluster management system}}
\item Sankel Software \textcolor{gray}{\textit{CAD/CAM, gaming and computer animation}}
\item Silk, Amsterdam \textcolor{gray}{\textit{Filter and visualize large amounts of information}}
\item Skedge Me \textcolor{gray}{\textit{Online scheduling platform}}
\item Standard Chartered \textcolor{gray}{\textit{Wholesale banking business}}
\item Starling Software, Tokio \textcolor{gray}{\textit{Commercial automated options trading system}}
\item Suite Solutions \textcolor{gray}{\textit{Management of large sets of technical documentation}}
}
\end{itemize}
{\small (Quelle: \url{http://www.haskell.org/haskellwiki/Haskell_in_industry})}
\end{frame}


%%%%%%%%%%%%%%%%%%%%%%%%%%%%%%%%%%%%%%%%%%%%%%%%%%
\begin{frame}[fragile]{\de{Bekannte funktionale Sprachen}\en{Well-known functional languages}}
\begin{tikzpicture}
\draw (1.5, 1.5) node {Lisp};
\draw (1.5, 5.5) node {Scheme};
\draw (4.2, 4.9) node {ML};
\draw (10.5, 3.5) node {OCaml};
\draw (1.5, 3.5) node {Miranda};
\draw (7.3, 4.6) node {F\#};
\draw (6.3, 5.6) node {Erlang};
\draw (9.3, 5.3) node {Clojure};
\draw (4.5, 1.5) node {Scala};

\onslide+<1-2>
\draw (5.5, 3.0) node {Haskell};
\onslide+<2>
\draw (8.5, 1.3) node {(JavaScript)};
\onslide+<3>
\draw (5.5, 3.0) node {\color{red}Haskell};
\draw (8.5, 1.3) node {\color{red}(JavaScript)};
\end{tikzpicture}

\end{frame}

%%%%%%%%%%%%%%%%%%%%%%%%%%%%%%%%%%%%%%%%%%%%%%%%%%
\begin{frame}[fragile]{\de{Was ist denn an funktionaler Programmierung so besonders?}\en{Now, what is so special about functional programming?}}

\onslide+<2->
\begin{center}
\Large
Immutability
\end{center}

\onslide+<3->
\begin{center}
\de{Jeder Variablen darf nur einmal ein Wert zugewiesen werden}
\en{Each variable can only be assigned to once}
\end{center}

\vspace{1em}

\onslide+<4->
\begin{center}
\Large
\de{Funktionen sind}\en{Functions are} \glqq{}first order citizens\grqq{}
\end{center}

\onslide+<5->
\begin{center}
\de{Mit Funktionen kann man dasselbe machen wie mit Strings oder Zahlen}
\en{Functions can be treated in the same way as strings or numbers}
\end{center}

\end{frame}

%%%%%%%%%%%%%%%%%%%%%%%%%%%%%%%%%%%%%%%%%%%%%%%%%%
\begin{frame}[fragile]{\de{Funktionen sind Werte}\en{Functions are values}}
\onslide+<2->
JavaScript:
\begin{lstlisting}{JavaScript}
function times(x, y) { return x * y; }

var timesVar = times;
    
timesVar(3, 5) === 15;
\end{lstlisting}

\onslide+<3->
Haskell:
\begin{lstlisting}{Haskell}
times x y = x * y

timesVar = times

timesVar 3 5 == 15
\end{lstlisting}

\end{frame}

%%%%%%%%%%%%%%%%%%%%%%%%%%%%%%%%%%%%%%%%%%%%%%%%%%
\begin{frame}[fragile]{\de{Funktionen sind Funktionsparameter}\en{Functions are function parameters}}
\onslide+<2->
JavaScript:
\begin{lstlisting}{javascript}
function times3(y) { return 3 * y; };

function apply(func, arg) { return func(arg); }

apply(times3, 5) === 15;
\end{lstlisting}

\onslide+<3->
Haskell:
\begin{lstlisting}{Haskell}
apply func arg = func arg

apply (\ x -> 3 * x) 5 == 15
\end{lstlisting}

\end{frame}

%%%%%%%%%%%%%%%%%%%%%%%%%%%%%%%%%%%%%%%%%%%%%%%%%%
\begin{frame}[fragile]{\de{Funktionen sind Rückgabewerte}\en{Functions are return values}}
\onslide+<2->
JavaScript:
\begin{lstlisting}{javascript}
function times(x) { return function (y) { return x * y; }; }

times(3)(5) === 15;
\end{lstlisting}

\onslide+<3->
Haskell:
\begin{lstlisting}{Haskell}
times x = (\y -> x * y)

times 3 5 == 15
\end{lstlisting}

\end{frame}

%%%%%%%%%%%%%%%%%%%%%%%%%%%%%%%%%%%%%%%%%%%%%%%%%%
\begin{frame}[fragile]{\de{Komisch, oder?}\en{Strange... ?!}}
JavaScript: \de{Zwei verschiedene Aufrufe}\en{Two different invocations}
\begin{lstlisting}{javascript}
function times(x, y) { return x * y; }
times(3, 5) === 15;

function times(x) { return function (y) { return x * y; }; }
times(3)(5) === 15;
\end{lstlisting}
~\\
Haskell: \de{Zweimal derselbe Aufruf}\en{Two identical invocations}
\begin{lstlisting}{Haskell}
times x y = x * y
times 3 5 == 15

times x = (\y -> x * y)
times 3 5 == 15
\end{lstlisting}

\end{frame}

%%%%%%%%%%%%%%%%%%%%%%%%%%%%%%%%%%%%%%%%%%%%%%%%%%
\begin{frame}[fragile]{Currying! (\de{oder auch Schönfinkeln}\en{also known as Schönfinkeling})}
\onslide+<2->
\de{In echten funktionalen Sprachen schreiben wir:}\en{In real functional languages we write:}
\begin{lstlisting}{Haskell}
times x y = x * y
\end{lstlisting}

\de{und eigentlich passiert Folgendes:}\en{but actually the following happens:}
\begin{lstlisting}{Haskell}
times x = (\y -> x * y)
\end{lstlisting}

\onslide+<3->
\de{Denn: Funktionen haben immer genau ein Argument}\en{Because functions always have exactly one argument}
\vfill
\onslide+<4->
\de{Nutzen: Partielle Evaluierung:}\en{Useful for partial evaluation:}
\begin{lstlisting}{Haskell}
times x y = x * y

times3 = times 3

times3 5 == 15
\end{lstlisting}


\end{frame}

%%%%%%%%%%%%%%%%%%%%%%%%%%%%%%%%%%%%%%%%%%%%%%%%%%
\begin{frame}[fragile]{\de{Und wenn ich kein Argument haben will?}\en{And what if I don't want an argument?}}
\begin{itemize}
\item \de{In echten funktionalen Sprachen bekommen Funktionen immer genau ein Argument!}\en{In real functional lanugages, functions always get exactly one argument!}
\item \de{Was ist, wenn ich nichts habe?!}\en{What if I don't want to pass anything?}
\onslide+<2->
\item \texttt{Unit} to the rescue!
\onslide+<3->
\item \texttt{Unit} \de{ist ein Typ mit nur einem Element}\en{is a type with only one element}
\onslide+<4->
\item \de{Das Element heißt in Haskell}\en{In Haskell, this element is called} \texttt{()}
\end{itemize}
\end{frame}

%%%%%%%%%%%%%%%%%%%%%%%%%%%%%%%%%%%%%%%%%%%%%%%%%%
\begin{frame}[fragile]{\de{Wichtige Bibliotheksfunktionen}\en{Important library functions}: filter}
\begin{itemize}
\item filter \de{oder auch}\en{or} select
\onslide+<2->
\item \de{Nimmt eine Collection und eine Funktion}\en{Takes a collection and a function}
\item \de{Liefert diejenigen Elemente der Collection, für die die Funktion \texttt{true} ergibt}\en{Returns those elements of the collection for which the function yields \texttt{true}}
\end{itemize}

\onslide+<3->
JavaScript: (\de{mit freundlicher Unterstützung der lodash-Bibliothek}\en{using the lodash-library})
\begin{lstlisting}{javascript}
_.filter([1, 2, 3, 4], function (x) { return x % 2 === 0; }) === [2, 4]
\end{lstlisting}

\onslide+<4->
Haskell:
\begin{lstlisting}{Haskell}
filter (\x -> x `mod` 2 == 0) [1,2,3,4] == [2,4]
\end{lstlisting}

\end{frame}

% find oder detect

%%%%%%%%%%%%%%%%%%%%%%%%%%%%%%%%%%%%%%%%%%%%%%%%%%
\begin{frame}[fragile]{\de{Wichtige Bibliotheksfunktionen}\en{Important library functions}: map}
\begin{itemize}
\item map \de{oder auch}\en{or} collect
\onslide+<2->
\item \de{Nimmt eine Collection und eine Funktion}\en{Takes a collection and a function}
\item \de{Liefert eine Collection, in der die Funktion auf jedes Element der ursprünglichen Collection angewandt wurde}\en{Yields a collection in which the function was applied to each element of the original collection}
\end{itemize}

\onslide+<3->
JavaScript:
\begin{lstlisting}{javascript}
_.map( [1, 2, 3, 4], function (x) { return x + 5; } ) === [6, 7, 8, 9]
\end{lstlisting}

\onslide+<4->
Haskell:
\begin{lstlisting}{Haskell}
map (\x -> x + 5) [1,2,3,4] == [6,7,8,9]
\end{lstlisting}

\end{frame}

%%%%%%%%%%%%%%%%%%%%%%%%%%%%%%%%%%%%%%%%%%%%%%%%%%
\begin{frame}[fragile]{\de{Wichtige Bibliotheksfunktionen}\en{Important library functions}: fold}
\begin{itemize}
\item fold \de{oder auch}\en{or} reduce \de{oder}\en{or} inject
\onslide*<2>{
\item \de{Nimmt eine Collection, eine Funktion und einen Startwert}\en{Takes a collection, a function and an initial value}
\item \de{Verbindet Startwert und erstes Element der Collection mit Hilfe der Funktion}\en{Merges initial value and first collection entry using the function}
\item \de{Verbindet das Ergebnis mit dem nächsten Element der Collection}\en{Merges the result and the next collection entry}
\item \de{Setzt dies für alle Elemente der Collection fort, bis nur noch ein Element übrigbleibt}\en{Continues for all collection entries, yielding a single result}
}
\end{itemize}

\onslide+<3->

JavaScript:
\begin{lstlisting}{javascript}
_.foldl( [2, 3, 4, 5], function (x, y) { return x * y; }, 1 ) === 120
\end{lstlisting}
\onslide+<4->
Haskell:
\begin{lstlisting}{Haskell}
foldl	(*) 1 [2,3,4,5] == 120
\end{lstlisting}

\onslide+<3->
\begin{center}
\includegraphics[width=.7\framewidth]{fold_1.png}
\end{center}

\end{frame}

%%%%%%%%%%%%%%%%%%%%%%%%%%%%%%%%%%%%%%%%%%%%%%%%%%
\begin{frame}[fragile]{\de{Typinferenz}\en{Type inference}}
\begin{itemize}
\item Haskell: \de{starkes statisches Typsystem}\en{strong static type system}
\item \de{Leichtgewichtige Verwendung dank Typinferenz}\en{Lightweight usage thanks to type inference}
\item \de{Herleitung des allgemeinst möglichen Typs}\en{Derivation of the most general type}
\end{itemize}

\onslide+<2->
\begin{lstlisting}{Haskell}
foldl :: (a -> b -> a) -> a -> [b] -> a
(*) :: Num a => a -> a -> a

foldl	(*) 1 [2,3,4,5]
\end{lstlisting}

\vfill

\onslide*<3>{
  \begin{center}
    \includegraphics[width=.7\framewidth]{fold_2.png}
  \end{center}
}

\onslide+<4->
\de{Compilerfehler für}\en{Compile error for}:
\begin{lstlisting}{Haskell}
foldl	(*) "x" [2,3,4,5]

No instance for (Num [Char]) arising from a use of `*'
Possible fix: add an instance declaration for (Num [Char])
\end{lstlisting}

\end{frame}


%%%%%%%%%%%%%%%%%%%%%%%%%%%%%%%%%%%%%%%%%%%%%%%%%%
\begin{frame}[fragile]{\de{Eine einfache Berechnung}\en{A simple calculation}}

\[
sum = \sum_{i=1}^{10}i^2
\]

\vspace{5em}

\onslide+<2->
\begin{lstlisting}{java}
int sum = 0;
for(int i = 1; i <= 10; i++) {
  sum = sum + i * i;
}
\end{lstlisting}
\end{frame}

%%%%%%%%%%%%%%%%%%%%%%%%%%%%%%%%%%%%%%%%%%%%%%%%%%
\begin{frame}[fragile]{\de{Exkurs}\en{Excursion}: Clean Code}
Single Responsibility Principle
\\[2em]

\onslide+<2->
\de{Wie viele Verantwortlichkeiten hat dieser Code?}\en{How many responsibilities does this code have?}
\begin{lstlisting}{java}
int sum = 0;
for(int i = 1; i <= 10; i++) {
  sum = sum + i * i;
}
\end{lstlisting}

\onslide+<3->
\begin{itemize}
\item \de{Erzeugen der Zahlenfolge von 1 bis 10}\en{Creating the sequence of numbers from 1 to 10}
\onslide+<4->
\item \de{Quadrieren einer Zahl}\en{Calculating the square of a number}
\onslide+<5->
\item \de{Berechnen der Quadratzahl jeder Zahl in der Folge}\en{Calculating the square of each number in the sequence}
\onslide+<6->
\item \de{Addieren zweier Zahlen}\en{Calculating the sum of two numbers}
\onslide+<7->
\item \de{Aufsummieren der Quadratzahlen}\en{Calculating the sum of all squares}
\end{itemize}

\end{frame}

%%%%%%%%%%%%%%%%%%%%%%%%%%%%%%%%%%%%%%%%%%%%%%%%%%
\begin{frame}[fragile]{\de{Trennen der Verantwortlichkeiten}\en{Separation of Concerns}}
\begin{itemize}
\item \de{Erzeugen der Zahlenfolge von 1 bis 10}\en{Creating the sequence of numbers from 1 to 10}
\onslide+<2->
\begin{lstlisting}{javascript}
var sequence = _.range(1, 11);
\end{lstlisting}
\onslide+<1->
\item \de{Quadrieren einer Zahl}\en{Calculating the square of a number}
\onslide+<3->
\begin{lstlisting}{javascript}
var square = function (i) { return i * i; };
\end{lstlisting}
\onslide+<1->
\item \de{Berechnen der Quadratzahl jeder Zahl in der Folge}\en{Calculating the square of each number in the sequence}
\onslide+<4->
\begin{lstlisting}{javascript}
var squaredSequence = _.map(sequence, square)
\end{lstlisting}
\onslide+<1->
\item \de{Addieren zweier Zahlen}\en{Calculating the sum of two numbers}
\onslide+<5->
\begin{lstlisting}{javascript}
var add = function (s1, s2) { return s1 + s2; };
\end{lstlisting}
\onslide+<1->
\item \de{Aufsummieren der Quadratzahlen}\en{Calculating the sum of all squares}
\onslide+<6->
\begin{lstlisting}{javascript}
var sum = _.reduce(squaredSequence, add);
\end{lstlisting}
\end{itemize}

\end{frame}

%%%%%%%%%%%%%%%%%%%%%%%%%%%%%%%%%%%%%%%%%%%%%%%%%%
\begin{frame}[fragile]{\de{Zusammensetzen der Komponenten}\en{Combining the components}}
JavaScript:
\begin{lstlisting}{javascript}
var square = function (i) { return i * i; };
var add = function (s1, s2) { return s1 + s2; };
\end{lstlisting}

\begin{lstlisting}{javascript}
_.reduce(_.map(_.range(1, 11), square), add) === 385;
\end{lstlisting}

\onslide+<2->
\de{oder}\en{or}

\begin{lstlisting}{javascript}
_(1).range(11).map(square).reduce(add) === 385;
\end{lstlisting}

\onslide+<3->
Haskell:
\begin{lstlisting}{Haskell}
foldl	(+) 0 (map (\x -> x*x) [1..10]) == 385
\end{lstlisting}
\de{oder}\en{or}
\begin{lstlisting}{Haskell}
(>.>) x f = f x
[1..10] >.> map (\x -> x*x) >.> foldl (+) 0 == 385
\end{lstlisting}

\end{frame}

%%%%%%%%%%%%%%%%%%%%%%%%%%%%%%%%%%%%%%%%%%%%%%%%%%
\begin{frame}[fragile]{\de{Uff!}\en{Phew!}}

\begin{center}
\Large
\de{OK, alle einmal tief durchatmen :-)}\en{OK, everybody take a deep breath :-)}
\end{center}

\end{frame}

%%%%%%%%%%%%%%%%%%%%%%%%%%%%%%%%%%%%%%%%%%%%%%%%%%
\begin{frame}[fragile]{Pattern Matching}
Fibonacci-\de{Funktion}\en{Function} \glqq \de{naiv}\en{na\"ive}\grqq{}:
\begin{lstlisting}{Haskell}
fib x = if x < 2 then x else fib (x-1) + fib (x-2)
\end{lstlisting}

\onslide+<2->
\vfill
Fibonacci-\de{Funktion mit}\en{Function with} Pattern Matching:
\begin{lstlisting}{Haskell}
fib 0 = 0
fib 1 = 1
fib x = fib (x-1) + fib (x-2)
\end{lstlisting}
\end{frame}

%%%%%%%%%%%%%%%%%%%%%%%%%%%%%%%%%%%%%%%%%%%%%%%%%%
\begin{frame}[fragile]{\de{Algebraische Datentypen}\en{Algebraic Datatypes}}
\de{Binärbaum}\en{Binary tree}:
\begin{lstlisting}{Haskell}
data Tree =
       Node Tree Tree
     | Leaf Int
\end{lstlisting}

\onslide+<2->
\begin{lstlisting}{Haskell}
myTree = Node (Node (Leaf 4) (Node (Leaf 7) (Leaf 1))) (Leaf 3)
\end{lstlisting}

\onslide*<2>{
\begin{center}
\includegraphics[width=.3\framewidth]{tree.png}
\end{center}
}

\onslide+<3->
\de{Summenfunktion}\en{Summary function}:
\onslide+<4->
\begin{lstlisting}{Haskell}
treeSum	(Leaf x) = x
\end{lstlisting}
\onslide+<5->
\vspace{-1.1em}
\begin{lstlisting}{Haskell}
treeSum (Node m n) = treeSum m + treeSum n
\end{lstlisting}
\onslide+<6->
\begin{lstlisting}{Haskell}
treeSum myTree == 15
\end{lstlisting}
\end{frame}


%%%%%%%%%%%%%%%%%%%%%%%%%%%%%%%%%%%%%%%%%%%%%%%%%%
\begin{frame}[fragile]{\de{Fazit}\en{Bottom line}}
\begin{itemize}
\item \de{Funktionale Programmierung ist verbreiteter als man denkt}\en{Functional programming is more common than you may have expected}
\item Java 8 \de{hat funktionale Neuerungen}\en{has new functional features}
\item \de{Viele Sprachen bringen funktionale Aspekte oder Zusatzmodule mit}\en{Many languages have functional aspects or additional modules}
\end{itemize}

\vfill

\onslide+<2->
\de{Referenzen}\en{References}:

\begin{itemize}
\item \de{Funktionale JS-Bibliothek}\en{Functional JS-Library}: \url{http://lodash.com}
\item Haskell: \url{http://www.haskell.org}
\end{itemize}

\end{frame}




%%%%%%%%%%%%%%%%%%%%%%%%%%%%%%%%%%%%%%%%%%%%%%%%%%
{
\begin{frame}{\de{Vielen Dank!}\en{Thank you very much!}}

        Code \& \de{Folien auf}\en{slides on} GitHub:
        \begin{center}
                \url{https://github.com} \url{/NicoleRauch/FunctionalProgrammingForBeginners}
        \end{center}

        ~\\[1em]
        \begin{block}{Nicole Rauch}
        \begin{description}[Twitterxx]
                \item[E-Mail]  \href{mailto:info@nicole-rauch.de}{\texttt{info@nicole-rauch.de}}
                \item[Twitter] \href{http://twitter.com/NicoleRauch}{\texttt{@NicoleRauch}}
                \item[Web] \href{http://www.nicole-rauch.de}{\texttt{http://www.nicole-rauch.de}}
        \end{description}
        \end{block}
\end{frame}
}
